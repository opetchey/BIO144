\documentclass[english,9pt,aspectraio=169]{beamer}
\usepackage{etex}
\usetheme{uzhneu-en-informal}
%\usepackage{uarial}
\usepackage[T1]{fontenc}
\usepackage[utf8]{inputenc}
\RequirePackage{graphicx,ae}
\usepackage{bm}
\usepackage{fancybox,amssymb,color}
\usepackage{pgfpages}
\usepackage{booktabs}
\usepackage{verbatim}
\usepackage{animate}
\usepackage{numprint}
\usepackage{vwcol} 
\usepackage{dsfont}
\usepackage{tikz}
\usepackage{amsmath,natbib}
\usepackage{mathbbol}
\usepackage{babel}
\usepackage{SweaveSlides}
\usepackage{multicol}
\usepackage{xcolor}


\usetheme{uzhneu-en-informal}
\DeclareMathOperator{\po}{Poisson}
\DeclareMathOperator{\G}{Gamma}
\DeclareMathOperator{\Be}{Beta}
\DeclareMathOperator{\logit}{logit}
\def\n{\mathop{\mathcal N}}

\definecolor{Gray}{RGB}{139,137,137}
\definecolor{darkred}{rgb}{0.8,0,0}
\definecolor{Green}{rgb}{0,0.8,0.3}
\definecolor{lightgreen}{rgb}{0,0.7,0.3}
\definecolor{Blue}{rgb}{0,0,1}
\def\myalert{\textcolor{darkred}}
\def\myref{\textcolor{Gray}}
\setbeamercovered{invisible}

\renewcommand{\baselinestretch}{1.2}
\beamertemplateballitem
\DeclareMathOperator{\cn}{cn} % Copy number
\DeclareMathOperator{\ccn}{ccn} % common copy number
\DeclareMathOperator{\p}{p} % common copy number
\DeclareMathOperator{\E}{E} % common copy number
\DeclareMathOperator{\given}{|} % common copy number
\def\given{\,|\,}
\def\na{\tt{NA}}
\def\nin{\noindent}
\pdfpageattr{/Group <</S /Transparency /I true /CS /DeviceRGB>>}
\def\eps{\varepsilon}

\renewcommand{\P}{\operatorname{\mathsf{Pr}}} % Wahrscheinlichkeitsmaß
\def\eps{\varepsilon}
\def\logit{\text{logit}}
%\newcommand{\E}{\mathsf{E}} % Erwartungswert
\newcommand{\Var}{\text{Var}} % Varianz
\newcommand{\Cov}{\text{Cov}} % Varianz
\newcommand{\NBin}{\text{NBin}}
\newcommand{\Po}{\text{Po}}
\newcommand{\N}{\mathsf{N}}

\newcommand{\hl}{\textcolor{red}}

\newcommand{\ball}[1]{\begin{pgfpicture}{-1ex}{-0.65ex}{1ex}{1ex}
\usebeamercolor[fg]{item projected}

{\pgftransformscale{1.75}\pgftext{\normalsize\pgfuseshading{bigsphere}}}
{\pgftransformshift{\pgfpoint{0pt}{0.5pt}}
\pgftext{\usebeamerfont*{item projected}{#1}}}
\end{pgfpicture}}%
\usepackage{multicol}
\newcommand{\ballsmall}[1]{\begin{pgfpicture}{-1ex}{-0.65ex}{.2ex}{.2ex}

{\pgftransformscale{1}\pgftext{\normalsize\pgfuseshading{bigsphere}}}
{\pgftransformshift{\pgfpoint{0pt}{0.5pt}}
\pgftext{\usebeamerfont*{item projected}{#1}}}
\end{pgfpicture}}%



\begin{document}

\fboxsep5pt

\frame{
\title[]{ \centering \Huge Kurs Bio144: \\
Datenanalyse in der Biologie}%\\[.3cm]
\author[Stefanie Muff, Owen L.\ Petchey]{\centering Stefanie Muff  \& Owen L.\ Petchey }
%\institute[]{Institute of Social and Preventive Medicine \\ Institute of Evolutionary Biology and Environmental Studies}
\date[]{Week 8: Interpretation, causality, cautionary notes \\ 27./28. April 2017}


\maketitle
}


\frame{\frametitle{Overview (todo: check)}
\begin{itemize}
\item Occam's razor principle.\\[2mm]
\end{itemize}

 
}


\frame{\frametitle{Course material covered today}
\begin{itemize}
\item todo
\end{itemize}

\vspace{4mm}
\textcolor{blue}{\bf Optional reading:}
\begin{itemize}
\item todo
\end{itemize}
}

\frame[containsverbatim]{\frametitle{Recap of Last week}
\begin{itemize}
\item todo
\end{itemize}
}

 
  


\frame{\frametitle{Occam's Razor principle}

Naive idea: 
Given a set of potential covariates, simply include all of them.\\[4mm]

However, this ignores the \alert{principle of {\bf parsimony}}:\\[4mm]

\colorbox{lightgray}{\begin{minipage}{10cm}
Systematic effects should be included in a model {\bf only} if there is convincing evidence for the need of them.
\end{minipage}}

\vspace{4mm}
\href{https://de.wikipedia.org/wiki/Ockhams_Rasiermesser}
{\beamergotobutton{See Wikipedia for ``Ockham's Rasiermesser''}}
}

 

\frame{\frametitle{Summary}

}
% \frame{References:
% \bibliographystyle{Chicago}
% \bibliography{refs}
% }



\end{document}
